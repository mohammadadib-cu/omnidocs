\chapter{Appendix: Derivation of Transport Equations}
This chapter explains the derivation of transport equations from first principles for mass, species, energy and soot variables for the reactors employed in omnisoot.
\section{Constant Volume Reactor}
\label{sec:derivconstuv}

\subsection{Continuity}
The rate change of mass of gas mixture is equal to the production rate of soot.

\begin{equation*}
	\frac{d}{dt}\left(m_{gas}\right)=V\left(1-\varphi\right)\sum_{i}{{\dot{s}}_iW_i},
\end{equation*}
where V is the reactor volume that stays constant during the process, and gas occupies a fraction of volume reactor without soot, so $\mathrm{m_{gas}= \rho V (1-\varphi)}$.

\begin{equation*}
	\frac{d}{dt}\left(\rho V(1-\varphi)\right)=V\left(1-\varphi\right)\sum_{i}{{\dot{s}}_iW_i}
\end{equation*}
\begin{equation*}
	\Downarrow
\end{equation*}
\begin{equation}
	\frac{d}{dt}\left(\rho (1-\varphi)\right)=\left(1-\varphi\right)\sum_{i}{{\dot{s}}_iW_i}
\end{equation}

\subsection{Species}

The rate change of mass of each species can be described as:

\begin{equation*}
	\frac{d}{dt}\left(m_k\right)=V\left(1-\varphi\right)\left({\dot{\omega}}_k+{\dot{s}}_k\right)W_k\ \ 
\end{equation*}
The mass of each species is defined as the total gas mass multiplied by the mass fraction as
\begin{equation*}
	\frac{dY_k}{dt}\rho\ V(1-\varphi)+Y_k\frac{d}{dt}\left(\rho V(1-\varphi)\right)=V\left(1-\varphi\right)\left({\dot{\omega}}_k+{\dot{s}}_k\right)W_k
\end{equation*}
\begin{equation*}
	\Downarrow \times 1/V
\end{equation*}
\begin{equation*}
	\frac{dY_k}{dt}\rho\ (1-\varphi)+Y_k\frac{d}{dt}\left(\rho (1-\varphi)\right)=\left(1-\varphi\right)\left({\dot{\omega}}_k+{\dot{s}}_k\right)W_k
\end{equation*}