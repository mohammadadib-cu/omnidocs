\chapter{Introduction}

\section{Motivation}
Carbonaceous nanoparticles are widely encountered in nature and engineering. Every year, nearly 9.5 megatons of soot (black carbon) is emitted into the atmosphere making it the third strongest contributor to climate change 

\section{Background}

Soot (black carbon) is a major air pollutant, and the third strongest contributor to climate change \cite{myhre2014anthropogenic}. Carbon black (CB) with similar synthesis process and properties to soot is the largest industrially produced nanomaterial by value and volume (15 megatons per year with a value of \$17B) with applications as a reinforcing agent in rubber \cite{international2017} and conductive additive in lithium-ion batteries \cite{Palomares2010}. Controlling properties of these particles is essential to determine functional properties of CB and environmental effects of soot, but it is challenging due to the complexity of soot formation, its coupling with gas phase chemistry \cite{Wang2011}, and fluid dynamic effects. So, accurate predictive models are required to describe fluid flow, gas chemistry and different steps of soot formation and evolution in a coupled manner.

Soot morphology can be described accurately by mesoscale simulations, such as Discrete Element Modeling (DEM) \cite{Kelesidis2017Flame}, but they are computationally expensive and difficult to interface with the computational fluid dynamic (CFD) methods. So, sectional population balance models (SPBM) coupled with power-laws are used in flow reactors \cite{naseri2022simulating} and laminar flames \cite{kholghy2016core}. However, their computational cost grows exponentially by increasing the number of sections.

Monodisperse population balance models (MPBM) are alternative modelling approaches that describe average particle properties by tracking their total concentration, mass and area \cite{Kholghy2021}, but their accuracy depends on the assumptions about particle morphology (e.g. approximating agglomerates as monodisperse and perfect spheres). However, when inception and surface growth are short \cite{Spicer2002} and high particle (number) concentrations are formed \cite{Kelesidis2017}, they lead to rapid attainment of self-preserving size distributions (SPSD) and agglomerates having asymptotic structure \cite{Goudeli2016}. In these conditions, a MPBM can achieve accuracies on par with DEM \cite{Kelesidis2017Flame}, SPBM \cite{Kelesidis2019} and experimental data \cite{Kholghy2021}.



